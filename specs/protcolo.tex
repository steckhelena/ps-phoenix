\documentclass{article}

\usepackage[utf8]{inputenc}
\usepackage[T1]{fontenc}
\usepackage[portuguese]{babel}
\usepackage{amsmath}
\usepackage{graphicx}
\usepackage{float}


\title{Protocolo de comunicação - Projeto em grupo equipe Phoenix}
\author{
    Luiz Eduardo Araujo Zucchi\\
    Marco Antonio Steck Filho
}
\date{\today}

\begin{document}
    \maketitle
    \newpage

    \section{Resumo}
        Neste documento definimos o protocolo de comunicação entre a placa externa e a interna do robô. Definimos também o mapeamento do controle para os comandos do robô.

    \section{Algorítimo}
        \label{sec:1}
        \paragraph{}
        O algorítimo de comunicação da placa externa com a interna se comportará do seguinte modo:
        \begin{enumerate}
            \item Estabelece conexão com o controle na placa externa.
            \item Loop infinito:
            \begin{enumerate}
                \item Ler dados do controle e identificar os botões relevantes pressionados.
                \item Se for botão analógico mapear a intensidade para a variável $\textit{intesity}$ o valor entre 0 e 255 e para a variavel \textit{sinal}(que indica o sinal antes do mapeamento).
                \item Transcrever os botões pressionados para os comandos estabelecidos na tabela \ref{tab:1}.
                \item Transmitir os comandos para a placa interna do robô via sinal de rádio de 2.4GHz por meio do comando enviaDados(int Size, char byte...).
            \end{enumerate}
        \end{enumerate}

        \begin{table}[H]
            \centering
            \caption{Mapeamento dos botões pressionados para os comandos do robô.}
            \label{tab:1}
            \begin{tabular}{|l|l|l|}
            \hline
            Botão              & Comando                                  & Chars  enviados             \\ \hline
            Analógico esquerdo & Controla potência do movimento           & {[}"m", intensity, sinal{]} \\
            Analógico direito  & Controle de direção                      & {[}"d", intensity, sinal{]} \\
            Botão X            & Altenar estado da arma(ligado/desligado) & {[}"w"{]}                   \\
            Botão R2           & Acelerar                                 & {[}"A"{]}                   \\
            Botão L2           & Reverso                                  & {[}"R"{]}                   \\ \hline
            \end{tabular}
        \end{table}

    \section{Ajustes dos analógicos}
        \paragraph{}
        Além do algorítimo descrito na seção \ref{sec:1} o software da placa externa deverá apresentar alguns ajustes, como a definição de uma \textit{"deadzone"} para o controle, ou seja, a definição de um nível de sensibilidade dos analógicos no qual a placa deve ignorar o que está recebendo, isso se deve à possibilidade de a posição de descanso destes analógicos não estar corretamente centralizada e, portanto, essa \textit{"deadzone"} promoverá um controle mais preciso sobre as ações do robô.

\end{document}
